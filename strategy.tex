\documentclass{article}
\usepackage{hyperref}
\usepackage[a4paper, margin=1in]{geometry}
\usepackage{xspace}
\usepackage{typed-checklist}

\newcommand\rfcword[1]{\emph{#1}\xspace}
\newcommand\must{\rfcword{must}}
\newcommand\mustnot{\rfcword{must not}}
\newcommand\required{\rfcword{required}}
\newcommand\shall{\rfcword{shall}}
\newcommand\shallnot{\rfcword{shall not}}
\newcommand\should{\rfcword{should}}
\newcommand\shouldnot{\rfcword{should not}}
\newcommand\recommended{\rfcword{recommended}}
\newcommand\may{\rfcword{may}}
\newcommand\optional{\rfcword{optional}}

\newcommand\filename[1]{\texttt{#1}\xspace}
\newcommand\readme{\filename{README}\xspace}
\newcommand\authorxml{\filename{author.xml}\xspace}
\newcommand\program[1]{\textsc{#1}\xspace}


\title{The TELOS Collaboration Approach to Reproducibility and Open Science}

\author{Ed Bennett\\for the TELOS Collaboration}

\date{2024-10-..., version 0.0.1}

\begin{document}

\maketitle

\abstract{
  The TELOS Collaboration is committed to producing and analysing data reproducibly,
  and sharing its research openly.
  In this document,
  we set out the ways that we make this happen,
  where there is scope for improvement,
  and how we plan to achieve this.
  This is intended to work both as a statement of policy,
  and a guide to practice for those beginning to work with us.
  Full tutorials on every aspect of reproducibility is
  beyond the scope of this document,
  but we refer to other resources for further information in these cases.
}

\makeatletter
\renewcommand\tableofcontents{%
    \subsection{\contentsname
        \@mkboth{%
           \MakeUppercase\contentsname}{\MakeUppercase\contentsname}}%
    \@starttoc{toc}%
  }
\makeatother

\section{About this document}

\subsection{Introduction}

Reproducibility and openness are becoming increasingly important in science.
Many fields have experienced some form of replication or reproducibility crisis,
where work that had been taken as fact was subsequently found
to be based on shaky results that did not stand up to scrutiny.
As a fully computational discipline,
there is no reason in principle
that work in lattice quantum field theory should not be fully reproducible end to end.
In some cases,
this comes into tension with making optimal use of computing resources;
in this document we will make explicit what compromises we make,
and where improvements could be made to reduce the impact of these.
It is also important that limited public resources not be wasted
repeating the same work in multiple contexts
because different groups did not have access to each others' data.
Our aim is to push forward the boundaries of human knowledge,
and the main barriers to this are human and computational capacity;
we believe that maximising the amount of our work that we share,
and the ability of others to access and build on top of it,
is the optimal way to overcome these.

\tableofcontents

\subsection{Version history}

\begin{description}
  \item[v0.0.1]
        Initial draft.
\end{description}

\subsection{Definitions}

\subsubsection{Reproducibility}\label{sec:reproducibility}

In our work,
we use the definitions of ``reproducibility'' and associated terms
used by the Turing Way project~\cite{the_turing_way_community_2023_7625728}.
Specifically,
we define an analysis as:
\begin{description}
  \item[reproducible]
        when another researcher is able to take the same data,
        perform the same analysis on it,
        and obtain the same results;
  \item[replicable]
        when another researcher is able to produce their own equivalent dataset,
        perform the same analysis on it,
        and obtain the same results;
        and
  \item[robust]
        when another researcher is able to take the same data,
        perform an alternative analysis on it,
        and obtain results leading to the same conclusions.
\end{description}
Reproducibility is the simplest of these to achieve,
and may be considered as trivial;
indeed,
science is predicated on it,
and so work that is not reproducible is of limited value.

\subsubsection{FAIR}

Data and software tools may be described as FAIR if they are
Findable, Accessible, Interoperable, and Reusable~\cite{wilkinson2016fair}.
Specifically,
data and metadata are
\begin{description}
  \item[Findable]
        if they have a unique, persistent identifier,
        are described by rich metadata linked back via this identifier,
        and are indexed or registered in a searchable resource;
  \item[Accessible]
        if they are retrievable using a standardised, open communications protocol,
        allowing for authentication and authorisation where necessary,
        and where metadata remain available even after associated data are not;
  \item[Interoperable]
        if they use a formal, accessible, shared, and broadly applicable language for knowledge representation;
        if they use vocabularies that follow the same FAIR principles,
        and use qualified references to other data and metadata;
        and
  \item[Reusable]
        if they are richly described with accurate and relevant attributes,
        released with a clear and accessible license,
        associated with detailed provenance,
        and meet community-relevant standards.
\end{description}

\subsubsection{Open Science}

Open Science or Open Research is
the movement to make all research accessible to all levels of society.
This may include not just papers,
but also other outputs like data, software, and experimental samples,
as well as including those from outside academia in
the process of conducting research itself.
This can only ever be an ideal to be strived towards,
being tensioned against the costs of opening access to limited resources.

\subsubsection{Copyright and licensing}

Under copyright law,
the default state is that
when you provide someone with a piece of copyrighted work
or a copy thereof,
they have no right to make further copies of it.
In a digital world,
this includes many operations that are essential for consuming content,
such as downloading a copy of a paper to read.
Some actions may be construed as ``fair dealing'' or ``fair use'',
where there is no penalty,
but many important actions
like reusing or building on top of a piece of work
require explicit permission.
All creative work,
including academic papers, data, and software code,
automatically receives copyright protection at the moment of its creation,
which typically lasts for seventy years beyond the death of its creator.

To avoid each person having to write their own forms of permission
(many of which would be found unenforceable on contact with a lawyer),
certain organisations have defined standard ``copyright licenses''
that may be applied to a piece of work
to grant others specific permissions regarding what they can do with it
and what constraints they must comply with to benefit from these permissions.

\paragraph{Creative Commons}

Creative Commons defines a number of licenses suitable for papers and data.
These can be recognised as a series of letters starting with ``CC''.
The most common,
and most relevant to our usage,
is the Creative Commons Attribution License,
CC BY~\cite{cc-by}.
This gives recipients the right to make and distribute copies of a work,
provided that the original author is clearly attributed.

\paragraph{Software Licenses}

Creative Commons licenses are not considered
suitable for licensing software source code,
due to concerns around software-specific issues such as patents.
Instead,
specific software licenses should be used.
Two common ones are the MIT License~\cite{mit},
which grants rights similar to those granted by CC BY,
and the GNU General Public License~\cite{gpl},
which imposes an additional requirement that
if derivative work is distributed,
this distribution is on the same terms.

Much modern gadgetry makes use of MIT- and similarly-licensed software,
with a long list of attributions,
but likely little support given back to the original authors.
GPL-licensed software is less used in this context,
due to organisations not wanting to share their proprietary software
and potentially give assistance to their competitors.
The aim of GPL is that
anyone should be free to modify and improve the software on their own devices;
however,
the extent to which this has succeeded is limited.

\subsubsection{Paper stages}

A paper has a number of stages in its preparation process:

\begin{description}
  \item[Draft]
        Version worked on by one person,
        or shared within the collaboration for comment
  \item[Preprint]
        Version posted on the arXiv for comments from the community,
        in advance of submission to a journal.
  \item[Author Accepted Manuscript]
        Last version submitted to a journal before acceptance.
        Incorporates feedback from peer reviewers.
  \item[Version of Record]
        Version distributed by the journal.
        In some cases
        (for example,
        in the Journal of High Energy Physics
        and Proceedings of Science)
        this will appear near-identical to the Author Accepted Manuscript;
        in others
        (such as in Physical Review D)
        this will have been
        translated by the editorial office into a different style,
        likely with one or more rounds of feedback
        and checking of proofs.
\end{description}

\subsubsection{Persistent identifiers}

A persistent identifier is
an identifier that can be used to refer to
a publication,
a data asset,
or some other object,
that is unique within some system,
and which provides some guarantee that it will remain available in the future.

The Digital Object Identifier,
or DOI,
is an example of a persistent identifier.
DOIs are issued by journals to refer to articles
and by arXiv for preprints,
and can also be generated for data and code assets.


\subsubsection{Keywords}

The key words
``\must'',
``\mustnot'',
``\required'',
``\shall'',
``\shallnot'',
``\should'',
``\shouldnot'',
``\recommended'',
``\may'',
and ``\optional''
in this document are to be interpreted as described in RFC 2119~\cite{rfc2119}.



\section{Publications}

\subsection{Open Access}

There has for a long time been a culture in theoretical particle physics,
including in lattice,
of sharing preprints openly using the arXiv~\cite{ginsparg2021lessons}.
More recently,
the SCOAP3 agreement has meant that
articles published in
typical journals for particle physics,
such as Physics Letters B and Physical Review D,
are made available as open access without additional charge.
However,
since not all journals or topics are included in this agreement,
and to ensure
that readers not able to afford journal subscriptions
are able to access the corrected versions of papers,
our funders now require us to retain rights to
``Author Accepted Manuscripts'',
that is,
the versions of papers as accepted by the journal,
after peer review,
so that they can be shared under an open license.

To enable this,
all papers submitted by the TELOS collaboration
\must include the following text in the submitted manuscript:

\begin{quote}
  For the purpose of open access,
  the authors have applied a Creative Commons attribution (CC BY) licence
  to any Author Accepted Manuscript version arising.
\end{quote}

Publishers sometimes remove this statement on publication;
this does not affect the process.

We \must also upload either a CC BY licensed Author Accepted Manuscript,
or a CC BY licensed Version of Record,
to an institutional repository.
Where an article has not been covered by SCOAP3,
we \mustnot upload a non-CC BY licensed Version of Record;
the Author Accepted Manuscript \must be uploaded in that case.


\subsection{\authorxml}

When the INSPIRE database service ingests papers from arXiv,
it typically must take the plain-text author list
and parse it into a structured collection of references to known or new authors.
This process is error-prone,
particularly for larger collaborations,
and for individuals whose names collide with common words associated with authorship.
(For example,
``Ed'' is both a name
and an abbreviation that may be placed in front of a name
to indicate that the latter's owner is an ``editor''.
This means that people with the former name frequently have their names abbreviated on INSPIRE.)

To avoid needing to manually send corrections to INSPIRE after each publication,
INSPIRE recommends providing
a structured, machine-readable set of metadata regarding authorship of a paper.
This should be done in the form of a file named \authorxml,
having a structure described in Ref.~\cite{inspire-authorxml}.

A template for this file is available in the TELOS Collaboration Resources repository~\cite{resources}.
If this is used,
elements marked \verb|TODO| \must be replaced,
and authors' details \must be checked,
before publication.

When preparing to submit to the arXiv,
in addition to the LaTeX sources and assets,
two additional files \should be included:

\begin{itemize}
  \item
        The file \authorxml,
        prepared as described above.
  \item
        The file \filename{author.dtd},
        defining the structure that \authorxml follows,
        available from Ref.~\cite{inspire-authorxml}.
\end{itemize}

\section{Data}\label{sec:data}

The work that we present in publications can represent
the output of hundreds of thousands of pounds' of computer time,
and terawatt hours of energy.
To ensure that this is not wasted,
and so that others can reproduce our work
and extract additional results beyond our initial work
without needing to spend similar resources to regenerate the data,
it is vital that we share our data openly.
This also enable those quoting our data to do so easily
without needing to transcribe numbers,
which may introduce errors.

Every peer-reviewed article that presents new data
\must have an associated data release,
including the new data that were generated in its preparation.
A work that generates no new data,
only presenting previously-analysed data,
\shouldnot have a data release.
Non-peer-reviewed work,
such as conference proceedings,
\may instead refer to
the data release of an upcoming peer-reviewed article
rather than having a separate data release;
however,
if such a work will not be forthcoming,
a data release \should be prepared and published.

A publication \must cite the associated data release using its DOI\@.
Where a publication makes use of data prepared for a previous publication,
the associated data release \must also be cited if it exists.
If the publication did not have an associated data release,
the data used from it \must instead by included in
the current publication's associated workflow release.

The data release associated with a publication
\should be published in advance of the preprint being made available,
and \must be published before the paper is published in the journal.

\subsection{What to include in a data release}

To maximise the utility of a data release to others,
a number of classes of data are needed.

\subsubsection{Final numbers}\label{sec:dr-numbers}

To maximise the utility of our results to those looking to make direct use of them,
data releases \must include all numbers that are plotted as points on a graph,
and \should include all parameters for curves fitted using lattice data
where these are of use to others.
These \must be provided in CSV format,
to maximise ease of use to those who may only be familiar with spreadsheet software.
The number of different CSV files \should be minimised:
data that are characterised by the same parameters
\should be combined into a single file.
For example,
there \may be one file for numbers relating to individual ensembles,
a second file for fit parameters relating to specific values of the coupling,
and a third for fit parameters for continuum limit extrapolations.

Currently there is no community-defined metadata schema for structuring such data.
Nevertheless,
we \should aim to use a common form for our data
to maximise interoperability with others' work.

\begin{itemize}
  \item
        Numbers with single symmetric uncertainties \must be formatted with columns labelled
        \verb|{name}_value| and \verb|{name}_uncertainty|.
  \item
        Numbers with asymmetric uncertainties \must use the suffixes
        \verb|upper_uncertainty| and \verb|lower_uncertainty|.
  \item
        Numbers with multiple sources of uncertainty \must use
        \verb|uncertainty| or \verb|statistical_uncertainty|
        as the suffix for the statistical uncertainty,
        and \must use suffixes of the form
        \verb|{uncertainty_type}_uncertainty|
        for other sources of uncertainty,
        such as systematic.
  \item
        Metrics associated with each other \must use a common prefix.
        For example,
        the mass and matrix element from fitting a pseudoscalar correlation function
        and the associated $\chi^{2}/\textnormal{d.o.f.}$
        might use column names
        \verb|ps_mass_value|,
        \verb|ps_mass_uncertainty|,
        \verb|ps_matrix_element_value|,
        \verb|ps_matrix_element_uncertainty|,
        and \verb|ps_chisquare|.
  \item
        Numbers \must be given to the full machine precision at which they were originally presented,
        not truncated or rounded.
  \item
        Column names \should be documented in the data release documentation
        (see Sec.~\ref{sec:dr-documentation} below).
\end{itemize}

\subsubsection{Input files}

To enable others to reproduce our work on HPC should they wish to,
a data release \may include the raw input files,
if they have been retained in a form that reproduces the original work.
If this is done,
the release \should also include sufficient documentation
to allow a competent practitioner to be able to use the input files to reproduce the data.

\subsubsection{Data from HPC}\label{sec:dr-from-hpc}

The TELOS Collaboration's workflow is typically that
some number of gauge ensembles are generated using an HMC-like algorithm,
taking significant computational resources
and generating substantial data.
Then,
observables are computed on each configuration in each ensemble,
still requiring HPC,
but more modest resources,
and generating outputs files small enough to process on a workstation.
These are then transferred off HPC for final statistical analysis.

Given the constraints of storage,
and to enable analysis workflows to be reproducible,
our data releases \must
share the observable results that are the inputs to the statistical analysis.
This \should be in as close to its native form as possible;
to date,
for most of our work,
this is text-based log files from \program{HiRep} and \program{Grid}.
Where necessary for filesize reasons,
these \may be thinned by removing redundant, repeated log lines.

To make the download process simpler for a reader only interested in a specific observable,
these \may be structured so that
each observable or class of observable
is in a separate directory hierarchy.


\subsubsection{Repackaged data}

Since it is tedious to write large amounts of code to parse data from log files,
and the reading process itself is relatively slow,
if the original output was in a text-based format,
a data release \should also include the input data to the statistical analysis
as discussed in Sec.~\ref{sec:dr-from-hpc} above,
reformatted into a packed format readable by standard libraries.

\begin{itemize}
  \item
        The HDF5 format~\cite{hdf5} \should be used for these data.
  \item
        A single HDF5 file for all ensembles \should be preferred
        to one file per ensemble per measurement,
        to minimise the effort needed to download and use the data.
  \item
        Datasets and groups in the HDF5 structure \must be given
        sufficient metadata to identify the specific ensembles they refer to,
        and what specific observables were computed and parameters used to do so.
\end{itemize}




\subsubsection{Metadata and analysis parameters}

As will be discussed in Section~\ref{sec:workflows} below,
to ensure that others are able to reproduce our work,
it is vital to share the workflows that are used to analyse our data.
These workflows will typically rely on knowing information about the data,
available in a more compact and quickly-readable form than
looking at the headers of each data file separately,
and will further rely on information that we choose as part of our analysis.
For example,
which ensembles do we consider for a particular analysis,
or where to we judge the plateau region of an observable to be for each ensemble.

This information is still data,
so belongs in the data release rather than being hidden mixed into the analysis workflow.

\begin{itemize}
  \item
        A data release \should contain one or more files containing metadata,
        and if appropriate,
        analysis parameters,
        for the analysis workflow to use as input.
  \item
        Metadata and analysis parameters \must be provided in a plain-text,
        human-readable format.
        This \may be a tabular format such as CSV,
        or a more structured format like YAML\@.
  \item
        Metadata and analysis parameters \should be consolidated into as few files as is reasonable,
        similarly to output numbers discussed in Section~\ref{sec:dr-numbers}.
  \item
        The structure of each metadata/parameter file,
        for example,
        field or column names,
        should be documented similarly to those for output numbers
        discussed in Section~\ref{sec:dr-numbers}.
  \item
        Metadata/parameter files \may be prepared using tools;
        for example,
        it is more convenient to prepare a large CSV file using a spreadsheet program
        rather than by hand in a text editor.
\end{itemize}


\subsubsection{Documentation}\label{sec:dr-documentation}

No matter how careful we are to construct our data releases carefully,
we will not be able to remove all ambiguity while maintaining a usefully compact release.
As such,
it is always necessary write documentation to enable others to understand our releases.

The TELOS Collaboration provides data release documentation in a file called \filename{README.md}
included in the data release.

\begin{itemize}
  \item
        A data release \must contain one or more files containing documentation.
  \item
        The primary documentation \should have a filename beginning \readme.
  \item
        Documentation \must be provided in a plain-text, human-readable format.
        At time of writing,
        this \should be Markdown.
  \item
        Where multiple documentation files are provided,
        each \must be cross-referenced from the main \readme.
  \item
        The main \readme \must provide the following elements:
        \begin{itemize}
          \item
                Information on the work the data were produced to enable,
                including a link to the article(s) in which they were presented.
          \item
                A listing of the files included in the release,
                and a description of what each file contains.
        \end{itemize}
  \item
        Where multiple files (or directories) are grouped into archive files,
        these archives \should contain their own \readme files
        documenting the data formats used for the files therein.
        Such files \must be referred to from the main \readme file.
  \item
        Descriptions of data formats, columns, etc.,
        \should be recycled from previous releases where appropriate.
        (If a release breaks compatibility with the format documented for a previous one,
        it \should be carefully considered whether this is necessary and justified,
        and \should be explicitly noted in the documentation.)
  \item
        Where single files are included in the data release,
        their structure \should be documented in the main \readme file.
  \item
        Portions of the main \readme file \may be used as
        the basis of the data release description.
\end{itemize}

\paragraph{Markdown}

For an introduction on how to write Markdown,
see for example Ref.~\cite{markdown-guide}.
For information on including mathematical expressions in Markdown documents,
see for example Ref.~\cite{github-markdown}.

\subsection{Where and how to publish a data release}\label{sec:publish-data}

\subsubsection{Where to publish}

Data releases \must be published using a platform that provides a persistent identifier
(preferably a DOI),
and that commits to maintaining availability of the dataset for an appropriate period.

At time of writing,
the TELOS Collaboration publishes its data releases using Zenodo~\cite{zenodo}.
Zenodo provides 50GiB of storage per dataset by default,
and allows up to 100 files per dataset.
(Zenodo can exceptionally grant up to 200GiB per dataset on a case-by-case basis.)
A release \mustnot be split into many Zenodo datasets in order to bypass this limit.

\subsubsection{Obtaining a DOI before data are ready}\label{sec:get-doi}

It is frequently desirable to have a DOI to put into a manuscript
before the data are ready to be uploaded,
either to avoid having a ``TODO'' item in the draft,
or because the data release is not ready at time of sharing a preprint.
In particular,
if a preprint is being released in advance of the data release being ready,
it \should refer to the DOI at which the data will ultimately be made available.

To obtain a DOI from Zenodo before the data are ready:

\begin{itemize}
  \item
        Start a new upload in Zenodo.
  \item
        Under ``Basic Information'', answer ``No'' to the question
        ``Do you already have a DOI for this upload?''.
  \item
        Select ``Get a DOI now!''
  \item
        Note the DOI that is generated,
        and add it to the references of the manuscript.
  \item
        Add a placeholder file
        (for example, the draft dataset \readme)
        and an author to the draft release.
  \item
        Click ``Save draft''
\end{itemize}

The saved draft can then be used as the basis for the full release once it is ready,
and citations to it from the draft or preprint will then be correctly resolved.

\subsubsection{Structuring the release}

Owing to the limitation on the number of files in a release,
and the lack of directory structure in Zenodo,
some files must be packaged into archives.
However,
to maximise the utility of the release to
those who may only want a small part of it,
thought must be given as to what to group together.

\begin{description}
  \item[The \readme]
        \must be a separate file.
  \item[CSV files]
        \should be uploaded as individual files,
        such that someone wanting to quote a datum from one of them
        need not download a lot of irrelevant files.
  \item[Raw data]
        \should be packaged into one or more archives.
        If there are multiple classes of data
        (for example,
        HMC logs, gradient flow logs, and correlation function logs),
        then these \may be packaged separately.
        Each archive \should produce a \filename{raw\_data} directory,
        containing a subdirectory for the specific observable,
        matching the structure expected by the analysis workflow.
        As discussed above,
        each archive \should contain its own \readme.
  \item[Repackaged raw data]
        \should be uploaded as a single large HDF5 file.
  \item[Metadata and analysis parameters]
        \may be packaged into one or more archives,
        or uploaded as single files,
        as appropriate.
  \item[Input files]
        \should be packaged into one or more archives.
\end{description}

Where archives are used,
these \should be ZIP files,
not \filename{.tar.gz} files,
since the former can be previewed by Zenodo.
They \should use a high compression level,
to reduce the storage and bandwidth requirements,
in particular for raw data comprising logs,
which are naturally storage-inefficient.

\subsubsection{Completing the upload form}
TODO

\subsection{Field configurations}

Field configurations take significant computational resources to generate,
and contain significantly more information than one person or collaboration can extract.
As such,
we aspire to share our field configurations openly,
to maximise the benefit that can be obtained from them.
(To this,
we apply the constraint that
configurations \shouldnot be retained if the cost of storage outweighs
the cost of regenerating them given the input parameters.)
In principle,
this would also allow data from different collaborations
to be analysed many of the same systematics,
rather than comparing data that have been computed and analysed with different approximations.

The International Lattice Data Grid (ILDG)~\cite{ildg-organization}
defines standards for interoperability and sharing of field configurations.
We aim to share our configurations using the UK Regional Grid of the ILDG\@.
Currently this is not in service;
further information on preparing and pushing configurations to this service
will be provided once it is available.

In the interim,
when generating ensembles requiring non-trivial computational effort,
we \must retain sufficient information that
the ILDG configuration and ensemble metadata may be completed at a later date.
This should include:

\begin{itemize}
  \item
        The gauge group
  \item
        The gauge action, associated parameters (e.g. $\beta$), and boundary conditions
  \item
        The fermion actions, associated parameters (e.g.\ bare mass), representations, and boundary conditions
  \item
        The algorithm used to perform the generation
  \item
        The software (including version information) used to perform the generation
  \item
        The precision worked at by the software used to perform the generation
  \item
        The name and affiliation of the person who performed the generation
  \item
        The date and time at which each configuration was generated
\end{itemize}


\section{Workflows}\label{sec:workflows}

In our data release,
we share data from various stages of our computation.
However,
we must also share how data get from one stage to the other.
The Royal Society shortly after its founding in 1660
chose as its motto ``\emph{nullius in verba}''---take nobody's word for it---
to signify the importance of this.
It marked a transition to science being based on reproducibility,
where no result could be accepted until others were able to show it for themselves.
This laid the foundation for the growth in science over the past centuries.

In principle,
sharing the data analysis code used to go from input to output data
is not a necessary condition for reproducibility.
A sufficiently precise,
and accurate,
narrative publication
could achieve the same result.
In practice,
however,
specifying the level of necessary detail
and keeping the actual software used synchronised with this
is beyond the capabilities of most authors;
in many fields this has led to a ``replication crisis''
where key findings in a discipline were discovered to be unfounded,
as the description of the methodology did not match what was actually done,
due either to human error in the analysis process,
or imprecise language in the description.

As such,
we assert that codifying the analysis performed in data and software,
and publishing both of these openly alongside narrative articles,
is the \emph{easiest} way to achieve reproducibility of our work.
Others looking to apply our techniques
(as they should,
if we are doing our jobs properly)
may make use of our code directly,
or may reimplement based on our descriptions,
but cross-check against our implementation for any discrepancies.

Working reproducibly has an initial learning curve,
and requires some more up-front setup than diving straight in to
fiddling with data and producing plots by hand.
However,
it pays substantial dividends as the quantity of data you work with increases,
the workflows become more complex and interlinked,
and the number of projects you work on grows.

In this section we will discuss some of the approaches that we adopt.
This will necessarily be incomplete,
and our approach is likely to evolve over time,
as experience shows us better ways of working,
and new tools and technologies become available.

\subsection{Workflow essentials}

All workflows \must be developed under version control.
In the TELOS Collaboration we use Git for this purpose;
if you are unfamiliar with Git,
good introductory guides include Refs.~\cite{swc-git,chacon2014pro}.
Workflows under development \should be regularly synchronised with a hosting service,
to avoid the only copy being on a laptop that may suffer data loss.
We make use of GitHub for this purpose.
Workflows \may be developed under the \texttt{telos-collaboration} organisation;
if not,
they \must be transferred to this organisation before publishing.

Each workflow repository \must contain the following files at root level:

\begin{itemize}
  \item A \readme file, for example \filename{README.md},
  \item A \filename{LICENSE} file, and
  \item A \filename{CITATION.cff} file.
\end{itemize}

It \should also contain the following:

\begin{itemize}
  \item A \filename{.gitignore} file, and
  \item A \filename{.pre-commit-config.yaml} file.
\end{itemize}

\subsubsection{\readme}

The \readme of a workflow release \must contain:

\begin{itemize}
  \item
        A brief description of the workflow,
        including a link to the article presenting the workflow's output.
        This \should be in the form of a DOI\@.
  \item
        A list of prerequisite software needed to be able to use the workflow.
        This is likely to include
        Conda,
        Snakemake
        (which may be installed via Conda),
        and a LaTeX distribution.
        This \should note any ``gotchas'' that may trip up a potential user,
        but \shouldnot give detailed installation instructions
        that would duplicate the documentation of the software in question.
  \item
        Instructions on setting up the workflow.
        This \should include how to clone the repository,
        what data to download from the data release,
        and where to put it.
  \item
        Instructions on running the workflow.
        This \should be a single command,
        and \must be equivalent to the command that was run
        to generate the assets presented in the corresponding article.
        (You \may,
        for example,
        change the parallelisation options to Snakemake is allowed,
        but \mustnot use an entirely different script.)
  \item
        Information on the approximate expected runtime of the workflow,
        and the machine specification on which this estimate is based.
        (It is useful for a reader to know in advance
        whether they will need to allocate
        minutes,
        hours,
        or days
        to the computation.)
  \item
        Information on where the output of the workflow is placed.
\end{itemize}

Additionally,
the \readme \should contain:

\begin{itemize}
  \item
        The DOI of the workflow;
        this \may be in the form of a DOI badge.
        To obtain a DOI from Zenodo prior to releasing it,
        see the instructions in Section~\ref{sec:get-doi}.
  \item
        A discussion of the reusability of the workflow.
        How much of the workflow is expected to be applicable to other contexts,
        and what work would be required to do so.
\end{itemize}

\subsubsection{\filename{LICENSE}}

The \filename{LICENSE} file should contain
the full text of the license under which the workflow is made available.
The TELOS Collaboration uses the GNU General Public License~\cite{gpl} by default.
If there are reasons to prefer another license,
this \must be discussed with the collaboration before applying it.

\subsubsection{\filename{CITATION.cff}}

To aid tools in generating appropriate citations for our work,
we provide metadata on the release in the form of a \filename{CITATION.cff} file.
This is written in the Citation File Format (CFF)~\cite{cff}.
CFF files may be generated using the \program{cffinit}~\cite{cffinit} tool,
or may be based on the skeleton example in the resources repository~\cite{resources}.
If the latter is used,
elements marked \verb|TODO| \must be replaced,
and authors' details \must be checked,
before publication.

\subsubsection{\filename{.gitignore}}

The \filename{.gitignore} file is a standard mechanism to tell Git
what files are
(typically)
not wanted to be committed to a repository.
This \may be based on a template such as that provided by GitHub~\cite{gitignore-python},
but \should also specify the directories expected to contain
data downloaded from the data release,
and files output by the workflow.
A minimal \filename{.gitignore} file for a workflow matching the description below might be,

\begin{verbatim}
# Common temporary files
scratch
.DS_Store
.#*
*#*#
*~
*__pycache__
**.pyc
**.ipynb_checkpoints
*.pdf
cache/
tmp/

# Input and output data
.snakemake/
raw_data/
intermediary_data/
data_assets/
assets/
metadata/
!*/.git_keep
\end{verbatim}

Including data files in the \filename{.gitignore} file early is important,
because if data files are accidentally committed to the repository,
then everyone who downloads the repository will need to download them:
even if a later commit removes the files again,
they will still be present in the history.
(It is possible to rewrite history to remove unwanted files,
but this takes additional work and requires coordination to not reintroduce them.)

\subsubsection{\filename{.pre-commit-config.yaml}}

We aim for our releases to be useful for others,
and to minimise the work required when we are developing them.
As such,
we try to make our code easy to read;
it is difficult enough to read one's own code
after a few weeks of not looking at it,
let alone someone else's.
To this end,
we use some basic automated code quality checks,
to keep the basic formatting of code consistent.

For this,
we make use of \program{pre-commit}~\cite{pre-commit}.
When inside a repository,
with \program{pre-commit} available in the \texttt{\$PATH},
running the command
\begin{verbatim}
pre-commit install
\end{verbatim}
adds a hook to the repository that is run when a commit is attempted;
it reads and runs the checks defined in \filename{.pre-commit-config.yaml},
and if any checks do not pass,
the commit is blocked.
Some checks automatically fix the files such that you can retry the commit quickly;
other checks may require manual fixes to be made.

A minimal \filename{.pre-commit-config.yaml} for our work might be:

\begin{verbatim}
default_language_version:
    python: python3.12
repos:
- repo: https://github.com/astral-sh/ruff-pre-commit
  rev: v0.5.1
  hooks:
    - id: ruff
      # Lint rules suggested by ruff docs, just for starters:
      args: [--fix]
    - id: ruff-format
- repo: https://github.com/pre-commit/pre-commit-hooks
  rev: v4.6.0
  hooks:
    - id: check-yaml
    - id: end-of-file-fixer
    - id: trailing-whitespace
    - id: check-toml
    - id: mixed-line-ending
- repo: https://github.com/jumanjihouse/pre-commit-hooks
  rev: 3.0.0
  hooks:
    - id: markdownlint
      files: "content/"
\end{verbatim}

The versions specified are correct at time of writing,
but are likely out of date by the time you read this document.
You \may use \program{pre-commit CI}~\cite{pre-commit-ci} to keep these up to date.

\subsection{Workflow management}

Performing an analysis in lattice quantum field theory has many moving parts.
We typically work with many ensembles,
for each of which we compute many observables.
We want to combine subsets of each of these in different ways,
and perform various fits of them.
Each analysis step takes a different amount of time,
and depends on others having been previously completed.

As discussed above,
we require that a single command be able to reproduce the full analysis,
end to end.
One way we might start thinking about that is by writing a shell script,
to run the various tools needed in order.
We might also consider using a single Python script
that calls the various functions that we need to run,
using subprocesses where we use non-Python tools.
However,
this comes with some limitations:
if we want to run multiple steps in parallel
(as we likely want to,
as the number of steps grows),
we need to manually specify how and where to parallelise.
And if we want to avoid recomputing
(potentially expensive)
steps that we have already completed
and whose inputs have not changed,
we need to manually perform these checks.

To avoid reinventing the wheel,
we instead choose to build on the work of others who have already solved these problems.
A workflow manager is a tool that allows you to specify
a set of rules for how to compute specific outputs given specific inputs;
when given a specific target to generate,
it identifies what steps needs to be performed,
and performs them.
A typical workflow manager does this
by building a directed acyclic graph
(DAG)
of the steps;
this then allows it to parallelise over all steps that do not directly depend on each other.

After a survey of the available options,
we identified \program{Snakemake}~\cite{molder2021sustainable} as a good fit for our typical needs.
It has a number of features that are useful to us:

\begin{itemize}
  \item
        It can run on one or multiple CPU cores.
  \item
        It can manage the software environments needed for running rules using Conda.
        No manual installation of environments is necessary for a user of the release,
        beyond installing Snakemake.
  \item
        It tracks what outputs it has generated using what inputs,
        so only re-runs rules when the input data change.
  \item
        The syntax for rule definitions is built on top of Python,
        so is extensible if the standard syntax does not meet our needs
\end{itemize}

It can also run parts of a workflow on external HPC resources;
this is not a feature we currently use,
but may be useful in future.

Currently,
there are no Snakemake tutorials specific to the lattice context;
while it is our ambition to change this,
for the time being,
we refer to e.g.\ Refs.~\cite{carpentries-snakemake,snakemake-tutorial}.

\subsubsection{Structuring a workflow}

The benefits of a workflow manager discussed above are maximised when
the structure of the workflow is aligned with the strengths of the workflow manager.
Put shortly:
\begin{itemize}
  \item
        The scope of each tool
        (or rule)
        should be kept as small as possible,
        with the complexity being encoded as relationships between rules in the workflow definition.
  \item
        Parameters should be known to the workflow manager,
        rather than passing a file containing them.
  \item
        Input and output files should be specified as command-line parameters,
        rather than being hard-coded.
  \item
        The workflow should not append to files,
        only create them afresh.
        If joining files together is required,
        that should be its own rule.
\end{itemize}

To take a concrete example,
if we have data files for meson correlation functions and gradient flow histories
at various values of the bare mass $am_{0}$,
and we wish to plot the masses of the pseudoscalar and vector mesons
$aM_{\mathrm{PS}}$ and $aM_{\mathrm{V}}$,
normalised by the gradient flow scale $w_{0}/a$
as a function of $am_{0}$,
then the workflow might be structured as the following:

\begin{itemize}
  \item
        A preamble,
        in which the workflow reads a metadata file with information about the ensembles,
        and in particular the plateau positions for each channel for each ensemble.
  \item
        A rule to take one ensemble's gradient flow history,
        and output bootstrap samples for $w_{0}$ to a file,
        labelled by an ensemble identifier and the observable name
        (for example,
        \texttt{Sp4nF2b6.7mF-0.62T48L32/\hspace{0pt}w0\_samples.json}).
        This file should also carry metadata and provenance information about the ensemble.
  \item
        A rule to take one ensemble's meson correlation function data,
        and using the plateau position metadata,
        compute the mass of one mesonic channel $aM_{X}$,
        outputting bootstrap samples to a file,
        labelled by an ensemble identifier and the channel name
        (for example,
        \texttt{Sp4nF2b6.7mF-0.62T48L32/\hspace{0pt}ps\_mass\_samples.json}).

        This file should also carry metadata and provenance information about the ensemble.
  \item
        A rule to take the bootstrap samples for $aM_{X}$ and $w_{0}/a$,
        and output the dimensionless product $\hat{M}_{X}=w_{0}M_{X}$,
        to a file labelled by an ensemble identifier and a description of the product
        (for example,
        \texttt{Sp4nF2b6.7mF-0.62T48L32/\hspace{0pt}normalised\_ps\_mass\_samples.json}).
        This file should also carry metadata and provenance information about the ensemble.
  \item
        A rule to take any number of files
        each containing the value of the normalised mass for a single ensemble,
        and a plot style definition,
        and output a plot of the normalised mass against the bare fermion mass
        (the latter obtained from the metadata).
\end{itemize}

To meet the requirements discussed in Section~\ref{sec:data} above,
there would also want to be a rule to take all of the various sample files
and output a CSV that could be included in the data release.

\subsection{Structuring the repository}

We have discussed above the essential files to include in the root directory of the repository.
Now,
let's go into more detail about how to structure the remainder of the repository.

\begin{itemize}
  \item
        Files relating to the definition of the workflow \should be placed in a \filename{workflow/} directory.
        The workflow itself \must be placed in \filename{workflow/Snakefile}.
        Workflows with many rules \should be split into modules;
        these \should be placed in \filename{workflow/rules/}.
        Conda environment definitions should be placed in \filename{workflow/envs/}.
        This structure matches the standard recommendations for Snakemake projects in general.
  \item
        Source files \should be placed in a \filename{src/} directory.
        For projects of any complexity,
        this \should be structured as a Python package,
        containing an \filename{\_\_init\_\_.py} file,
        so that relative imports can be made.
  \item
        Definitions of,
        for example,
        plot styles \should be placed in a \filename{styles/} directory.
  \item
        Libraries that are not available via standard package repositories
        \should be placed in a \filename{libs/} directory,
        as discussed below.
  \item
        Input data \should be placed in a \filename{raw\_data/} directory.
        This directory \may contain a \filename{.git\_keep} file
        so that it is created when the repository is cloned,
        but all other files in the directory \must be ignored in \filename{.gitignore},
        to avoid accidentally committing large amounts of data.
  \item
        Data that have been quoted from other publications
        that did not provide a data release
        (so where numbers had to be transcribed by hand)
        \should be placed in a \filename{quoted\_data/} directory.
        They \must include documentation as to their provenance,
        including attribution to the original authors.
  \item
        Files containing metadata and analysis parameters
        \should be placed in a \filename{metadata/} directory.
        This directory \may contain a \filename{.git\_keep} file,
        but all other files in the directory \must be ignored in \filename{.gitignore}.
  \item
        Data produced by the workflow but not intended for distribution
        (for example,
        the bootstrap sample files discussed above)
        \should be placed in an \filename{intermediary\_data/} directory.
        This directory \mustnot contain a \filename{.git\_keep} file,
        and \must be ignored in \filename{.gitignore}.
  \item
        Data produced by the workflow and intended for distribution
        (for example,
        CSV files of final numbers to include in the data release)
        \should be placed in a \filename{data\_assets/} directory.
        This directory \may contain a \filename{.git\_keep} file,
        particularly if it is anticipated that users will place files into the directory by hand,
        but all other files in the directory \must be ignored in \filename{.gitignore},
        to avoid accidentally committing large amounts of data.
  \item
        Outputs generated by the workflow for inclusion in a manuscript
        (plots, tables, and variable definitions)
        \should be placed into an \filename{assets/} directory.
        This directory \may contain a \filename{.git\_keep} file,
        but all other files in the directory \must be ignored in \filename{.gitignore}.
        When preparing a manuscript,
        the directory \should be copied directly into the project;
        when re-running the workflow,
        the previous version \should be deleted from the project
        before the updated version is copied in.
\end{itemize}


\subsubsection{Libraries}

For published libraries,
it is typically sufficient to allow Conda to install them from standard repositories,
such as PyPI for Python packages.
For libraries only made available via GitHub or similar forge services,
these \mustnot be specified via a GitHub
(or similar)
URL,
for the same reason we do not use GitHub to publish our workflows---GitHub
does not provide a guarantee of long-term stability.

Instead,
where libraries are internal and/or sufficiently actively developed that
they are not available via a package repository,
these \must be included as Git submodules,
and installed directly from the local copy.
We place local copies of libraries in a \filename{libs} directory.

To download a repository as a submodule,
use the syntax
\begin{verbatim}
mkdir -p libs
cd libs
git submodule add https://github.com/username/reponame
\end{verbatim}
This addition can then be committed to the repository as usual.
Note that the URL used \must be a publicly-accessible HTTPS address,
not a private repository or a \texttt{git@github.com:} SSH address.

To clone a repository that includes submodules,
use
\begin{verbatim}
git clone --recurse-submodules git@github.com:telos-collaboration/workflow_name
\end{verbatim}
When adding a submodule to a repository for the first time,
the \readme should be updated to include this instruction in place of
the plain \texttt{git clone} that may otherwise be present.

To specify a local copy of a library in a Conda environment specification,
replace the \verb|library_name==0.0.1| or similar specification
generated by \texttt{conda env export}
with \verb|../../libs/library_name|.

Note that when GitHub exports a ZIP file of a repository,
it does not include the contents of any submodules.
As such,
we must create our own ZIP files of such repositories when uploading to Zenodo.

\subsection{Assets to generate}

In general,
the workflow should generate four broad classes of output:
plots,
tables,
definitions,
and data assets.

When naming files,
the workflow \mustnot generate files with the same name in different directories.
This is because arXiv only partially supports subdirectories;
while it can compile projects with files in subdirectories,
if two files in different subdirectories have the same name,
neither is accessible to LaTeX,
and the compilation fails.

\subsubsection{Plots}

Plot generation is one of the first tasks researchers approach.
A full deep dive into automated plotting is outside the scope of this document,
but we will highlight some key points specific to the way that we work.

\begin{itemize}
  \item
        Plots \should be produced using Matplotlib,
        or a tool based on its engine.
  \item
        Plots \should read in style information from a common style file,
        and \shouldnot have excessive cosmetic overrides.
  \item
        The workflow \should make it simple to change which plot style file is in use,
        for example,
        to switch to using sans-serif fonts and a dark background
        for presentation slides.
  \item
        Plots \should use a standard colour cycle,
        to maximise accessibility for those with colourblindness.
  \item
        Similarly,
        where possible,
        plots \should use differing markers in addition to colours.
        Where necessary,
        plots \may use markers and colours to differentiate different degrees of freedom.
  \item
        Plots \should use the \verb|layout="constrained"| option to \verb|plt.figure|
        (or to \verb|plt.subplots|)
        to maximise the use of space.
  \item
        Plots \should specify a \verb|figsize|
        equal to the anticipated size in the final manuscript,
        so that they may be used in LaTeX without specifying the \verb|\width| option.
        This ensures that font sizes remain consistent and legible.
  \item
        Plots \should be included in LaTeX files using syntax like:
\begin{verbatim}
\includegraphics{assets/plots/spectrum}
\end{verbatim}
        The \texttt{width=} argument \shouldnot be specified,
        and should not need to be.
  \item
        Axis and other labels on plots \must match the notation used in the main text.
  \item
        Where the caption of a figure depends on the specific parameters used to generate it,
        then a \filename{.tex} file \should also be generated containing the caption.
        For example,
        if the workflow generates a plot \filename{spectrum\_beta2.3.pdf},
        it should also generate a file \filename{spectrum\_beta2.3\_figure.tex},
        containing something along the lines of
\begin{verbatim}
\begin{figure}
  \includegraphics{assets/plots/spectrum_beta2.3}
  \caption{\label{fig:spectrum-betatwopointthree}
  The spectrum of the theory at $\beta=2.3''}
\end{figure}
\end{verbatim}
        In this instance,
        the figure \should be included in the manuscript using
\begin{verbatim}
\input{assets/plots/spectrum_beta2.3_figure.tex}
\end{verbatim}
  \item
        Plots \should be placed in an \filename{assets/plots/} directory.
\end{itemize}

\subsubsection{Tables}

Tables are frequently constructed by hand,
or generated using tools but then copied and pasted into a manuscript.
This manual intervention is error prone,
and increases the likelihood that the work presented will not be reproducible,
due to numbers from different iterations of the workflow
(or other numbers entirely,
from typographical errors)
being mixed into the final manuscript.

\begin{itemize}
  \item
        There \must be a 1:1 correspondence between tables and \filename{.tex} files generated.
        That is,
        each table should be in its own \filename{.tex} file.
  \item
        Each table file \must contain the entire table,
        starting with \verb|\begin{tabular}|
        (or equivalent)
        and ending with \verb|\end{tabular}|
        (or equivalent).
  \item
        Where the caption of the table depends on data presented therein,
        the workflow \should also include the caption in the \filename{.tex} file.
        The file \should then contain the entire \verb|table| environment,
        starting with \verb|\begin{table}|
        and ending with \verb|\end{table}|.
  \item
        Table files \must be included in the manuscript using commands along the lines of
\begin{verbatim}
\input{assets/tables/spectrum_table.tex}
\end{verbatim}
  \item
        Tables generated by a single workflow
        \must follow the same formatting conventions,
        in particular around where to place vertical and horizontal lines.
        Where possible,
        this \should be achieved by making use of common functions,
        allowing style changes to be made by changing a single definition.
  \item
        Table definitions \should be preceded by metadata describing them and their provenance.
        This \must be formatted as LaTeX comments
        (i.e.\ lines beginning with \verb|%|).
  \item
        Tables \should be placed in an \filename{assets/tables/} directory.
\end{itemize}

\paragraph{How to generate tables programmatically}

For plain tables of numbers without uncertainties,
Pandas DataFrames have a \verb|styler.to_latex| method to enable this.
For tables of numbers with uncertainties,
this method may be used in conjunction with
the \program{format\_multiple\_errors}~\cite{fme} library.
Where horizontal lines are needed to break up sections of a table,
currently this is challenging to achieve with Pandas,
but a pull request is open that may enable this in future.

\subsubsection{Definitions}

Frequently,
we want to discuss numerical results in our papers,
including quoting the numbers themselves.
Our first temptation may be to write the number in the text manually.
However,
if during drafting,
an additional data point is added that changes a fit result in the second decimal place,
it is easy for these numbers to become inconsistent with the actual analysis results.

To avoid this,
our workflows \should output definitions of LaTeX commands
that may be used in documents as placeholders for the numbers,
that will then be filled in by the workflow.
For example,
our workflow may define a command
\begin{verbatim}
\newcommand\PSMassContinuum{0.1234(56)(78)}
\end{verbatim}
which would then be used in the text as
\begin{verbatim}
We find the mass of the pseudoscalar in the continuum limit to be \PSMassContinuum.
\end{verbatim}
which would in turn render in the document as\\
``We find the mass of the pseudoscalar in the continuum limit to be 0.1234(56)(78).''

\begin{itemize}
  \item
        Definitions \should be produced for all numbers presented with uncertainties in a manuscript.
  \item
        Where definitions have been produced,
        they \must be used in place of all instances of the relevant number.
  \item
        Definitions \shouldnot force math mode,
        since in some instances they may be used inside a larger mathematical environment.
        (They \may use \verb|\ensuremath|.)
  \item
        Definition files \must have the \filename{.tex} extension.
  \item
        Definition files \should be placed in an \filename{assets/definitions/} directory.
  \item
        Each definition file \should begin with metadata describing the definitions and their provenance.
        This \must be formatted as LaTeX comments
        (i.e.\ lines beginning with \verb|%|).
  \item
        Definition files \must be loaded in the preamble of the manuscript LaTeX document,
        along the lines
\begin{verbatim}
\input{assets/definitions/spectrum_definitions.tex}
\end{verbatim}
\end{itemize}

\subsection{Metadata and provenance tracking}

It is useful
when files are taken outside of their original context
to be able to identify where they came from.
A full realisation of this would require implementing something like PROV~\cite{prov};
this is beyond the scope of our ambitions.
However,
retaining a subset of provenance information still has value.
This may be ``stamped'' as a comment in a file,
and/or included as an additional file.

\begin{itemize}
  \item
        When storing intermediary results,
        particularly when JSON or similar formats are used,
        files \should also store structured metadata and provenance information.
  \item
        When producing tables and definition files,
        these \should also include
        unstructured metadata and provenance information
        in LaTeX comments.
  \item
        When producing figures as PDF files,
        including provenance information is non-trivial,
        and is currently considered out of scope.
        When producing other formats
        (for example, Scalable Vector Graphics),
        provenance information \may be included as a comment.
  \item
        When producing directories of output
        (for example, \filename{assets/}),
        there \should be a machine-readable
        (for example, JSON)
        file placed at its root
        indicating metadata and provenance specific to the workflow run that produced it.
\end{itemize}

Things to include in a provenance stamp will vary from case to case,
but \should most likely include:

\begin{itemize}
  \item
        A comment warning that the asset was/assets were automatically generated,
        and should not be directly modified,
        but instead that the workflow should be re-run.
        This \should be sorted to appear at the top of the listing.
  \item
        The commit ID of the workflow used to generate the asset,
        and if there were uncommitted changes present.
  \item
        The time at which the asset was generated.
  \item
        The computer using which the asset was generated.
  \item
        The person who ran the workflow.
        (For example, their username.)
  \item
        For assets generated from other data,
        the files that were used as input.
  \item
        Parameters that were used in the generation of the asset.
\end{itemize}


\subsection{Numerical reproducibility}

While in Sec.~\ref{sec:reproducibility} we define reproducibility in black and white,
there are in fact certain degrees to which data are reproducible.
They may give compatible results within error bars,
or may give identical results down to the last bit,
or something in between.

An analysis workflow \should give bitwise identical output,
except for differences in headers and other provenance data,
when run on the same platform,
with the same parallelisation options.
It \must give compatible results,
with differences much smaller\footnote{
  We deliberately do not explicitly define ``much smaller'' at this time.
} than the statistical error,
when re-run,
including running on other machines or with different parallelisations.

\subsubsection{Randomness}

Analysis workflows frequently make use of randomness.
In particular,
bootstrap sampling always uses randomness,
and for a finite number of bootstrap samples,
the final numbers obtained will have
an in principle negligible but noticeable dependence on the choice of random numbers.
As such,
we \must fix random seeds before performing a computation with randomness.
This \must be done separately for each workflow rule;
since the workflow may execute in any order compatible with the data dependencies,
different runs may otherwise give different results.

Random seeds \must be different for each ensemble,
since otherwise correlations are introduced between them.
In order to avoid hardcoding specific random seeds,
which may raise questions in the minds of readers of our workflows
(``Where did that number come from?''
``Was it chosen to give a specific outcome?''),
we generate seeds deterministically from input metadata.
To introduce sufficient entropy into these,
it is \recommended to use a hash of the ensemble name as the random seed.
This has the benefit that
having the same bootstrap samples for different observables is simplified.

An example Python implementation may look like:

\begin{verbatim}
import hashlib
import numpy as np

def get_rng(ensemble_name):
    ensemble_hash = hashlib.md5(ensemble_name.encode("utf8")).digest()
    seed = abs(int.from_bytes(filename_hash, "big"))
    return np.random.default_rng(seed)
\end{verbatim}

This approach will give results that are reproducible to near-machine precision.



\subsection{Publishing a workflow}

It is not sufficient to link to a GitHub repository from a journal article.
GitHub provides no guarantees of long-term stability of repositories,
or even of the addresses to them.
Instead,
similarly to data releases,
we \must publish workflows using a service that provides a persistent identifier,
and a commitment to long-term availability.

At time of writing,
the TELOS Collaboration publishes its workflows using Zenodo~\cite{zenodo}.

The procedure for this is largely the same as that used for data releases,
described in Sec.~\ref{sec:publish-data}.
Differences for workflow releases are discussed below.

\subsubsection{Preparing an archive}

In principle,
one may connect GitHub with Zenodo,
and automatically generate a record from a GitHub Release.
However,
as discussed above,
this does not include the contents of submodules,
meaning that the archives downloaded from the release would not be usable.

To ensure that all submodules are correctly included in the release,
we \should clone a fresh copy of the repository,
and archive it,
keeping only the working copy,
i.e. excluding the \filename{.git} directories:

\begin{verbatim}
mkdir tmp
cd tmp
git clone --recurse-submodules git@github.com:telos-collaboration/workflow_name
zip -9 --exclude "**/.git/*" --exclude "**/.git"  -r workflow_name workflow_name
\end{verbatim}

One \may instead work from the existing working copy,
but in this case they \must first strip out all data not in the repository,
for example using \verb|git clean|.
Data from the data release \mustnot be included in the workflow release.

\subsubsection{What to include in a release}

The release should include two files:

\begin{itemize}
  \item
        The \readme file from the repository,
        which \should also form the basis of the Zenodo description, and
  \item
        The ZIP archive of the repository.
\end{itemize}



\section{HPC Software}
TODO


\bibliographystyle{unsrt}
\bibliography{references}

\clearpage
\appendix

\section{Checklist for publishing a journal article}

In this Appendix is a checklist that \should be completed for all publications.
Underneath each high-level goal
(circles)
is a set of tasks that \should be completed before the goal can be completed
(squares).

\begin{CheckList}{Goal}
  \Goal{open}{Circulate draft to collaboration}
  \begin{CheckList}{Task}
    \Task{open}{DOI for data release obtained and added to draft}
    \Task{open}{DOI for analysis release obtained and added to draft}
    \Task{open}{Hand-generated provisional/placeholder plots marked}
    \Task{open}{Software used is committed to version control}
    \Task{open}{Software names and versions (e.g.\ commit IDs) are specified}
    \Task{open}{Rights retention statement is present}
    \Task{open}{Data availability statement is present}
  \end{CheckList}
  \Goal{open}{Publish data release}
  \begin{CheckList}{Task}
    \Task{open}{\readme contains data format descriptions for all files}
    \Task{open}{Raw data are all included}
    \Task{open}{CSVs are present for all data presented in paper (tabulated or plotted)}
    \Task{open}{Archives do not contain unwanted files (e.g.\ operating system metadata files like \filename{.DS\_Store})}
    \Task{open}{Release cross-checked with analysis workflow by another collaboration member}
  \end{CheckList}
  \Goal{open}{Publish analysis workflow}
  \begin{CheckList}{Task}
    \Task{open}{\readme includes requirements installation instructions}
    \Task{open}{\readme includes instructions for getting input data}
    \Task{open}{\readme includes instructions for running workflow}
    \Task{open}{\readme notes expected run time of workflow}
    \Task{open}{
      No data are hardcoded into code.
      (Check for any numbers with more than three significant figures.)
    }
    \Task{open}{Quoted data carry appropriate attribution}
    \Task{open}{Software environment is specified in one or more \filename{.yml} files}
    \Task{open}{Workflow runs end to end from a single command without errors}
    \Task{open}{
      Workflow archive does not include any unwanted files
      (e.g.\ data files, or operating system metadata files like \filename{.DS\_Store})
    }
    \Task{open}{Workflow cross-checked with data release by another collaboration member}
  \end{CheckList}
  \Goal{open}{Publish pre-print on arXiv}
  \begin{CheckList}{Task}
    \Task{open}{\authorxml written and included in source package}
    \Task{open}{Data release public \emph{(optional)}}
    \Task{open}{Analysis workflow release public \emph{(optional)}}
    \Task{open}{Final analysis workflow is committed to version control}
    \Task{open}{
      All plots, tables, and quoted numbers generated from workflow.
      No placeholder markers remain.
    }
    \Task{open}{All assets regenerated from a clean workflow run}
  \end{CheckList}
  \Goal{open}{Submit manuscript to journal}
  \begin{CheckList}{Task}
    \Task{open}{Preprint added to website}
    \Task{open}{arXiv paper password forwarded to coauthors}
  \end{CheckList}
  \Goal{open}{Submit corrected manuscript to journal after implementing referee feedback}
  \begin{CheckList}{Task}
    \Goal{open}{
      If changes have been made to analysis or presentation,
      all assets regenerated from a clean workflow run.
    }
    \Task{open}{Updated analysis workflow is committed to version control}
  \end{CheckList}
  \Goal{open}{Proofs are approved with journal, article is published}
  \begin{CheckList}{Task}
    \Task{open}{Author Accepted Manuscript uploaded to institutional repository (e.g.\ Cronfa)}
    \Task{open}{Finalised data release public \emph{(required)}}
    \Task{open}{Finalised analysis workflow release public \emph{(required)}}
    \Task{open}{
      If updated data/workflow releases published,
      manuscript updated to point to DOI of updated version.
    }
  \end{CheckList}
  \Goal{open}{Publication process finalised}
  \begin{CheckList}{Task}
    \Task{open}{Website updated with published article}
    \Task{open}{Institutional repository (e.g.\ Cronfa) updated with journal details}
  \end{CheckList}
\end{CheckList}
\end{document}
